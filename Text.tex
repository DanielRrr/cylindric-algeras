\documentclass{article}
\usepackage{graphicx} % Required for inserting images
\usepackage{amsthm}
\usepackage{amsmath}
\usepackage{amsfonts}
\usepackage{amsopn}
\usepackage{comment}
\usepackage{amssymb}

\theoremstyle{defin}
\newtheorem{definition}{Definition}

\theoremstyle{theorem}
\newtheorem{theorem}{Theorem}

\theoremstyle{claim}
\newtheorem{claim}{Claim}

\theoremstyle{prop}
\newtheorem{prop}{Proposition}

\theoremstyle{lemma}
\newtheorem{lemma}{Lemma}

\theoremstyle{fact}
\newtheorem{fact}{Fact}

\theoremstyle{remark}
\newtheorem{remark}{Remark}

\theoremstyle{ex}
\newtheorem{ex}{Example}

\theoremstyle{col}
\newtheorem{col}{Corollary}

\theoremstyle{question}
\newtheorem{question}{Question}

\let\strokeL\L
\renewcommand\L{\mathbf{L}}

\title{Cylindric notes}
\author{Daniel Rogozin}
\date{May 2023}

\begin{document}

\maketitle

\nocite{*}




\section{Cylindric algebras: background}

\subsection{Atomic representations of Boolean algebras}

Let $B$ be a Boolean algebra, an element $a$ is an \emph{atom} if for every $b \in B$
$b < a$ implies $b = 0$. $\operatorname{At}(B)$ is the set of all atoms
of $B$.

\begin{definition}~\label{compeleterep}
Let $B$ be a Boolean algebra and $F$ a field of sets such that $h : B \to F$ is a
representation of $B$, then $B$ is a complete representation of $B$, if for every
$A \subseteq B$ such that $\Sigma A$ is defined the following holds:
\begin{equation}
  h(\Sigma A) = \bigcup \limits_{a \in A} h(a)
\end{equation}
\end{definition}

A representation $h$ is called atomic, if $x \in h(1)$, then there exists 
$b \in \operatorname{At}(\mathcal{B})$ such that $x \in h(b)$.

\begin{theorem}~\label{completeboolean}
  Let $\mathcal{B}$ be a Boolean algebra, then $\mathcal{B}$ is atomic iff 
  $\mathcal{B}$ is completely representable. See~\cite[Corollary 6]{hirsch1997complete}.
\end{theorem}

\subsection{Proper cylindric algebras}

Let $X \neq \emptyset$ along and $X^{\omega} = \{ f \: | \: f : \omega \to X \}$.
Let $x \in X^{\omega}$, $x_i$ stands for $x(i)$ for $i < \omega$. 
A subset of $X^{\omega}$ is an $\omega$-ry relation on $U$. 
For $i, j < \omega$, the \emph{$i,j$-diagonal} $D_{ij}$ is the set of all elements of $X^{\omega}$ 
such that $y_i = y_j$.

If $i < \omega$ and $Y$ is an $\omega$-ry relation on $X$, then
the $i$-th cylindrification $C_i Y$ is the set of all elements of $U$ that agree with some element of $Y$ 
on each coordinate except, perhaps, the $i$-th one:
\begin{center}
$C_i Y = \{ y \in X^{\omega} \: |
\: \exists x \in Y \forall i < \omega \: (i \neq j \Rightarrow y_j = x_j)\}$.
\end{center}
We define the following equivalence relation for $i < \alpha$ and $x, y \in X^{\omega}$:
\begin{center}
 $x \equiv_i y \Leftrightarrow \forall j \in \alpha \: (i \neq j \Rightarrow x(i) = y(j))$
\end{center}
Then one may reformulate the definition of the $i$-th cylindrification the following way:

\begin{center}
 $C_i Y = \{ y \in X^{\omega} \: | \: \exists x \in X \:\: x \equiv_i y \}$
\end{center}

According to this version of the definition, one can think of cylindrification operators as ${\bf S5}$ modalities.

\begin{definition}
 A cylindic set algebra of dimension $\omega$ is an algebra consisting of a set $S$ of $\omega$-ry 
 relation on some base set $X$ with the constants and operations $0 = \emptyset$, $1 = X^{\omega}$, 
 $\cap$, $-$, the diagonal elements ${(D_{ij})}_{i, j < \omega}$, 
 the cylindrifications ${(C_i)}_{i < \omega}$. 
 A \emph{generalised cylindric set algebra of dimension $\omega$} is a subdirect of cylindric algebras
 that have dimension $\omega$.
  ${\bf Cs}_{\omega}$ denotes the class of all cylindric set algebras of dimension $\omega$.
\end{definition}

\begin{definition}
   A cylindric algebra of dimension $\\omega$ is an algebra $C = (B,  {(c_i)}_{i < \omega}, {(d_{ij})}_{i, j < \omega})$ 
   such that $B$ is a Boolean algebra, each $d_{ij} \in \mathcal{B}$ and for all $i, j, k < \omega$ and for all $a, b \in B$:
   \begin{enumerate}
     \item $c_i 0 = 0$,
     \item $c_i (a + b) = c_i a + c_i b$,
     \item $a \leq c_i a$,
     \item $c_i (a \cdot c_i b) = c_i a \cdot c_i b$
     \item $d_{ii} = 1$,
     \item $c_i c_j a = c_j c_i a$,
     \item If $k \neq i, j$, then $d_{ij} = c_k (d_{ij} \cdot d_{jk})$,
     \item If $i \neq j$, then $c_i (d_{ij} \cdot a) \cdot c_i (d_{ij} \cdot - a) = 0$.
   \end{enumerate}
   ${\bf CA}_{\omega}$ is the class of all cylindric algebras of dimension $\omega$.
\end{definition}

One can define a representation of a cylindric algebra explicitly the following way:

\begin{definition}
 Let $\mathcal{A}$ be a cylindric algebra of dimension $\omega$. 
 A \emph{representation} of $\mathcal{A}$ over the non-empty domain $X$ 
 is a one-to-one map $f : \mathcal{A} \hookrightarrow 2^{X^{\omega}}$ such that:
 \begin{enumerate}
   \item $f(1) = \bigcup \limits_{i \in I} X_i^{\omega}$ for some disjoint family ${\{X_i\}}_{i \in I}$ where each 
   $X_i \subseteq X$
   \item $h : \mathcal{A} \to 2^{f(1)}$ is a representation of a Boolean reduct
   \item for all $i, k < \omega$, $x \in h(d_{ik})$ iff $x_{i} = x_{k}$
   \item for all $i < \omega$ and $a \in \mathcal{A}$, $x \in h(c_{i}(a))$ 
   iff there is $y \in X$ such that $x[i \mapsto y] \in h(a)$
 \end{enumerate}
\end{definition}

Let $C$ be a cylindric algebra, $C$ is \emph{representable} 
if there exists a representation of $C$. ${\bf RCA}_{\omega}$ 
is the class of all representable cylindric algebras. 
Alternatively, ${\bf RCA}_{\omega}$ can be defined as the closure of ${\bf Cs}_{\omega}$ under isomorphism:
\begin{center}
${\bf RCA}_{\omega} = {\bf ICs}_{\omega}$.
\end{center}

It is well known that ${\bf RCA}_{\omega}$ is a variety, ${\bf RCA}_n$ is finitely axiomatisable for $n \leq 2$ and 
${\bf RCA}_{\alpha}$ ($2 < \alpha < \omega$) has no finite axiomatisation, 
see~\cite{Henkin1988-HENCAP-4}.

Let $C \in {\bf RCA}_{\omega}$, $\mathcal{A}$ has a \emph{complete representation} if its representation 
preserves all existing suprema as in Definition~\ref{compeleterep}.
In other words, $\mathcal{A}$ is \emph{completely representable}.

\begin{prop}
Let $A \in {\bf CA}_{\omega}$, then $A$ is completely representation iff it has an atomic representation.
\end{prop}

\begin{proof}
Follows from Theorem~\ref{completeboolean}.
\end{proof}

\section{Atom structures and canonical extensions}

First of all, we introduce the following operations on classes of algebras or frames. 
Let $\mathcal{A}$ be a class of algebras and $\mathcal{F}$ a class of frames, then:
\begin{itemize}
\item ${\bf I} \mathcal{K}$ is the closure of $\mathcal{K}$ under isomorphic copies,
\item ${\bf Ud} \mathcal{F}$ is the closure of $\mathcal{F}$ under disjoint unions,
\item ${\bf Ub} \mathcal{F}$ is the closure of $\mathcal{F}$ under bounded unions,
\item $\mathcal{F}^+$ is the class of all complex algebras generated from elements of $\mathcal{F}$,
\item ${\bf Pu} \mathcal{K}$ is the closure of $\mathcal{K}$ under ultraproducts,
\item ${\bf Pw} \mathcal{K}$ is the closure of $\mathcal{K}$ under ultrapowers,
\item ${\bf S} \mathcal{A}$ is the closure of $\mathcal{A}$ under subalgebras,
\item $\mathbb{S} \mathcal{F}$ is the closure of $\mathcal{F}$ under generated subframes,
\item $\mathbb{H} \mathcal{F}$ is the closure $\mathcal{F}$ under $p$-morphic images.
\end{itemize}

The following definition of an $\omega$-frame is taken from~\cite{Venema2013}.
\begin{definition}
  A cylindric $\omega$-frame is a structure $F = (W, {(R_i)}_{i < \omega}, {(D_{ij})}_{i, j < \omega})$ 
  where ${(R_i)}_{i < \omega}$ are binary relations and ${(D_{ij})}_{i, j < \omega}$ 
  are unary relations such that, for all $i, j, k < \omega$:
  \begin{enumerate}
  \item Every $R_i$ is an equivalence relation on $W$,
  \item $R_i \circ R_j = R_j \circ R_i$,
  \item For all $x \in W$, $D_{ii} = W$.
  \item For all $x, y, z \in W$, if $x R_i y$, $x R_i z$, $D_{ij} = W$ and $D_{ij} = W$, then $y = z$.
  \item For all $x \in W$, $D_{ij} = W$ iff there exists $y \in W$ such that $x R_k y$, $D_{ik} = W$ and $D_{kj} = W$.
  \end{enumerate}
  ${\bf CFrames}_{\omega}$ is the class of all $\omega$-frames.
\end{definition}

\begin{remark}

$ $

Observe that the conditions of cylindric $\omega$-frames can be expressed as first-order formulas. 
Therefore, ${\bf CFrames}_{\omega}$ is an elementary class.
\end{remark}

We can associate a complete atomic cylindric algebra of dimension $\omega$ with every cylindric $\omega$-frame 
$F = (W, {(R_i)}_{i < \omega}, {(D_{ij})}_{i, j < \omega})$ by taking its \emph{complex algebra}, 
which is the algebra $F^{+} = (2^{W}, \cup, -, {(C_i)}_{i < \omega}, \emptyset, W, {(D_{ij})}_{i, j < \omega})$ 
where each $C_i$ is an operator $C_i : 2^W \to 2^W$ defined as:
\begin{center}
$C_i A = \{ w \in W \: | \: \exists a \in A \: w R a \} = R_i^{-1}(A)$.
\end{center}

If $F \in {\bf CFrames}_{\omega}$ and $x \in F$, then $F^{x}$ is a \emph{generated subframe} generated by $x$. 
Generally, $F_1$ is a generated subframe of $F_2$, if $W_1 \subseteq W_2$, ${R_i}_1 \subseteq {R_i}_2$ 
and for all $x \in W_1$ $y \in R_{i_2}(x)$ implies $y \in W_1$ for every $i < \omega$. 
That is, for all $i < \omega$ and $x \in F_1$, we have $R_{i_2}(x) \subseteq F_1$ and, thus, 
$R_{i_1}(x) = R_{i_2}(x)$.

Let $F_1 = (W_1, {(R_{i_1})}_{i < \omega}, {(D_{{ij}_1})}_{i, j < \omega})$ and $F_2 = (W_2, {(R_{i_2})}_{i < \omega}, {(D_{ij_2})}_{i, j < \omega})$ be cylindric $\omega$-frames. 
A \emph{bounded morphism} is a function $f : F_1 \to F_2$ such that, for each $i, j < \omega$:
\begin{enumerate}
\item (Monotonicity) $x R_{i_1} y$ implies $f(x) R_{i_2} f(y)$ for all $x, y \in W_1$,
\item (The lifting property) If $f(x) R_{i_2} z$, then there exists $y \in R_{i_1}(x)$ such that $f(y) = z$,
\item $x \in D_{{ij}_1}$ iff $f(x) \in D_{{ij}_2}$.
\end{enumerate}
A bounded morphism is a \emph{$p$-morphism} if it is onto. Notation: $F_1 \twoheadrightarrow F_2$. 
In this case, we say that $F_1$ is a \emph{$p$-morphic image} of $F_2$. 

We have the following connection between $\omega$-frames and their generated subframes, which is standard for modal logic:
\begin{prop}
Let $F \in {\bf CFrames}_{\omega}$, then
\begin{enumerate}
\item $F = \coprod \limits_{x \in F} F^{x}$,
\item $F^+ \cong \prod \limits_{x \in F} {(F^{x})}^+$,
\item ${(F^{x})}^+$ is subdirectly irreducible.
\end{enumerate}
\end{prop}

Let $F$ be a cylindric $\omega$-frame and let ${(F_j)}_{j \in J}$ be a family of cylindric 
$\omega$-frames such that each $F_j$ is a generated subframe of $F$. 
Then $G = (W, R_i, D_{ij})$ is the \emph{bounded union} of ${(F_j)}_{j \in J}$, 
where $W = \bigcup \limits_{j \in J} W_j$ and $R_i$ and $D_{ij}$ are defined by corresponding relations in $F_j$'s.

The following fact connects cylindric frames and cylindric algebras through complex algebras, see~\cite[Proposition 2.1.5]{Venema2013}:
\begin{prop}~\label{framealgebra}
A structure $F$ is a cylindric $\omega$-frame iff $F^{+}$ is a cylindric algebra of dimension $\omega$.
\end{prop}

Let ${(F_j)}_{j \in J}$ be a disjoint family of cylindric $\omega$-frames, the \emph{disjoint sum} of 
${(F_i)}_{i \in I}$ is $F = \coprod \limits_{i \in I} F_i$, 
where each $R_i = \bigcup \limits_{j \in J} {R_i}_j$ and 
$D_{ik} = \bigcup \limits_{j \in J} {D_{ik}}_j$. 
Disjoint sums and direct products are connected with one another through complex algebras as follows 
(see~\cite[Lemma 3.4.1]{goldblatt1989varieties}):
\begin{equation} \label{prod}
{(\coprod \limits_{j \in J} F_j)}^{+} \cong \prod \limits_{j \in J} F_j^{+}
\end{equation}


\vspace{\baselineskip}

We define a particular frame of cylindric $\omega$-frames. 
Let $X$ be a non-empty set, the \emph{full Cartesian structure over $X$ of dimension $\omega$} 
is a cylindric $\omega$-frame 
$\mathfrak{C}(X) = (X^{\omega}, {(\equiv_i)}_{i < \omega}, {D_{ij}}_{i, j < \omega})$. 
$\mathcal{F}\mathfrak{ct}_{\omega}$ is the class of all full Cartesian structures of dimension $\omega$. 
Observe that
\begin{equation} \label{cmcubes}
{\bf Cs}_{\omega} = {(\mathcal{F}\mathfrak{ct}_{\omega})}^{+},
\end{equation}
\begin{equation} \label{isocmcubes}
{\bf I Cs}_{\omega} = {\bf S}{(\mathcal{F}\mathfrak{ct}_{\omega})}^{+}.
\end{equation}

The class of \emph{generalised cylindric set algebras} of dimension $\omega$, ${\bf Gs}_{\omega}$, 
consists of complex algebras of the closure of $\mathcal{F}\mathfrak{ct}_{\omega}$ under disjoint unions:
\begin{equation} \label{Gsone}
{\bf Gs}_{\omega} = {({\bf Ud}(\mathcal{F}\mathfrak{ct}_{\omega}))}^{+}
\end{equation}
or, by~(\ref{prod}):
\begin{equation} \label{Gstwo}
{\bf Gs}_{\omega} = {\bf P} (\mathcal{F}\mathfrak{ct}_{\omega}^{+})
\end{equation}
${\bf RCA}_{\omega}$ is the closure of ${\bf Gs}_{\omega}$ under isomorphism:
\begin{equation}
{\bf RCA}_{\omega} = {\bf IGs}_{\omega}
\end{equation}
or, assuming (\ref{Gsone}) and (\ref{Gstwo}):
\begin{equation}
{\bf RCA}_{\omega} = {\bf IGs}_{\omega} = {\bf S} ({({\bf Ud} (\mathcal{F}\mathfrak{ct}_{\omega}))}^{+}) = {\bf S} {\bf P} (\mathcal{F}\mathfrak{ct}_{\omega}^{+}).
\end{equation}

\vspace{\baselineskip}

If $C \in {\bf CA}_{\omega}$ is atomic, then we can associate a cylindric omega frame with it. 
Let $C$ be an atomic cylindric algebra of dimension $\omega$, its \emph{atom structure} 
is the structure ${\bf At}(C) = (\operatorname{At}(C), {(R_i)}_{i < \omega}, {(D_{ij})}_{i, j < \omega})$ 
such that each $D_{ij} \subseteq {\bf At}(C)$ and for all $i < \omega$ and 
for all $a, b \in \operatorname{At}(C)$:
\begin{center}
$a R_i b$ iff $c_i b \leq a$.
\end{center}

As a corollary from Proposition~\ref{framealgebra}:

\begin{prop}
If $C \in {\bf CA}_{\omega}$ is atomic, then ${\bf At}(C)$ is a cylindric $\omega$-frame.
\end{prop}

\section{Canonical extensions}

Let $B$ be a Boolean algebra, a proper subset $F \subsetneq B$ is an \emph{filter} if the following holds:
\begin{enumerate}
\item $a \in B$ and $a \leq b$ imply $b \in B$,
\item If $a, b \in B$, then $a \cdot b \in B$.
\end{enumerate}
A filter $U$ is an \emph{ultrafilter} if either $a \in U$ or $- a \in U$, or, equivalently, 
$U \subseteq U'$ implies $U = U'$. ${\bf Spec}(B)$ is the \emph{spectum} of $B$, 
that is, the set of all ultrafilters of $B$.

Let $C$ be a cylindric algebra of dimension $\omega$, the ultrafilter frame of $C$ is a structure 
$C_+ = ({\bf Spec}(C), {(R_i)}_{i < \omega}, {(D_{ij})}_{i, j < \omega})$ such that, 
for all $U_1, U_2 \in {\bf Spec}(C)$ and for all $i, j < \omega$:
\begin{enumerate}
\item $U_1 R_i U_2$ iff $\{ c_i a \: | \: a \in U_2 \} \subseteq U_1$,
\item $D_{ij} \subseteq {\bf Spec}(C)$.
\end{enumerate}

From Proposition~\ref{framealgebra} we have:
\begin{prop}
If $C$ is a cylindric algebra, then $C_+$ is a cylindric $\omega$-frame.
\end{prop}

\emph{The canonical extension} of $C$ is the algebra ${(C_+)}^+$, that is, the complex algebra of the ultrafilter frame.

\begin{theorem} (See \cite{jonsson1951boolean})

$C \in {\bf CA}_{\omega}$ embeds to ${(C_+)}^+$ by mapping 
$a \mapsto \{ U \in {\bf Spec}(C) \: | \: a \in U \}$.
\end{theorem}

\section{Canonicity of ${\bf RCA}_{\omega}$}

In this section, we reproduce the results related to characterisation ${\bf RCA}_{\omega}$. 
The following results are due to Goldblatt~\cite{goldblatt1995elementary}. 
This denotes that a cylindric algebra of dimension algebra is representable iff 
it is isomorphic to a subalgebra of the complex algebra of disjoint sum of some full 
$\omega$-dimensional Cartesian structure.

\vspace{\baselineskip}

The following characterisation result is known from~\cite[Theorem 2.2.3]{Venema2013}.

\begin{theorem}
${\bf RCA}_{\omega} = {\bf HSP}(\mathcal{F}\mathfrak{ct}_{\omega}^{+})$
\end{theorem}

That is, the class of representable cylindric algebras of dimension $\omega$ is a variety generated by complex algebras of full Cartesian structures of dimension $\omega$. If we consider the equational theory of ${\bf RCA}_{\omega}$ as a polymodal logic, we could say that it is Kripke complete with respect to the class of all full Cartesian structures of dimension $\omega$.

To show that ${\bf RCA}_{\omega}$ is canonical we have got to show the following inclusion:

\begin{center}
${({{\bf RCA}_{\omega}}_{+})}^+ \subseteq {\bf RCA}_{\omega}$.
\end{center}


\begin{definition}
The weak Cartesian space with base set $X$ and dimension $\omega$ determined by $x \in X^{\omega}$ is the set:
\begin{center}
${X^{\omega}}^{(x)} = \{ y \in X^{\omega} \: | \: \operatorname{card}(\{ k < \omega \: | \: x_k \neq y_k \}) < \aleph_0 \}$
\end{center}
$\mathfrak{S}_{\omega}({X^{\omega}}^{(x)})$ is a weak Cartesian structure of dimension $\omega$. 
$\mathcal{W}\mathfrak{ct}_{\omega}$ is the class of all weak Cartesian structure of dimension $\omega$ up to isomorphism.
\end{definition}
Note that we have $\mathcal{W}\mathfrak{ct}_{\omega} \subseteq {\bf CFrames}_{\omega}$.

Define also the class $\mathcal{S}\mathfrak{ct}_{\omega}$ of \emph{sub-Cartesian structures of dimension $\omega$} 
consisting of $\mathfrak{S}_{\omega}(V)$ for $V \subseteq X^{\omega}$, 
where $X$ is a non-empty base set. 
Note that $\mathfrak{S}_{\omega}(V)$ does not have to be a cylindric $\omega$-frame.

Let $F$ be a generated subframe of a full Cartesian structure of dimension $\omega$ $\mathfrak{C}(X)$, then
\begin{equation}
F \cong \coprod \limits_{x \in F} F^x
\end{equation}
or, by (\ref{prod}):
\begin{equation}
F^{+} \cong \prod \limits_{x \in F} {(F^{x})}^{+}
\end{equation}
The latter implies the inclusion:
\begin{equation}
{(\mathbb{S} \mathcal{F}\mathfrak{ct}_{\omega})}^+ \subseteq {\bf P} (\mathcal{W}\mathfrak{ct}_{\omega}^{+}).
\end{equation}

Note that (follows from~\cite[p. 118]{henkin1981cylindric}):
\begin{fact}
$\mathcal{W}\mathfrak{ct}_{\omega}^+ \subseteq {\bf RCA}_{\omega}$
\end{fact}

Complex algebras based on $\mathfrak{S}_{\omega}(X^x)$ form the class ${\bf Ws}_{\omega}$ of \emph{weak cylindric set algebras of dimension $\omega$}. The class ${\bf Gws}_{\omega}$ of \emph{generalised weak cylindric set algebras of dimension $\omega$} consists of complex algebras based on the closure of ${\bf Ws}_{\omega}$ under disjoint unions:
\begin{equation}
{\bf IWs}_{\omega} = {\bf S} (\mathcal{W}\mathfrak{ct}_{\omega}^{+})
\end{equation}

\begin{equation}
{\bf IGws}_{\omega} = {\bf S} ({({\bf Ud} \mathcal{W}\mathfrak{ct}_{\omega})}^+) = {\bf S P} (\mathcal{W}\mathfrak{ct}_{\omega}^{+})
\end{equation}

The following is by Goldblatt, see~\cite[Lemma 3.4]{goldblatt1995elementary}:
\begin{lemma}
${\bf RCA}_{\omega} = {\bf S} \: ({(\mathbb{S} {\bf Ud} \mathcal{F}\mathfrak{ct}_{\omega})}^{+}) = {\bf S} \: ((\mathbb{S} {\bf Ud} (\mathcal{W}\mathfrak{ct}_{\omega}))^{+}) = {\bf IGws}_{\omega}$
\end{lemma}

\subsection{Ultraproducts of full Cartesian structures}

Let ${(F_j)}_{j \in J}$ be an indexed family of full Cartesian structures of dimension $\omega$, 
where each $F_j$ is of the form 
\begin{center}
$F_j = (W_j, {(R_{i_j})}_{i < \omega}, {(D_{{ik}_j})}_{i, k < \omega})$
\end{center}
and let $U$ be an ultrafilter on $U$. 
Define the following equivalence relation on 
$\prod \limits_{j \in J} W_j$ for $f, g \in \prod \limits_{j \in J} W_j$:
\begin{center}
$f \sim_{U} g$ iff $\{ j \in J \: | \: f(j) = g(j) \} \in J$
\end{center}
The \emph{ultraproduct} of ${(F_j)}_{j \in J}$ is an algebra 
$\prod \limits_{J} F_j / U = (W, {(R_i)}_{i < \omega}, {(D_{ik})}_{i, k < \omega})$, 
where $W = \prod \limits_{j \in J} W_j$ and
\begin{enumerate}
\item $f_U R_i g_U$ iff $\{ j \in J \: | \: R_{i_j}(f_U(j), g_U(j)) \} \in U$,
\item $f_U \in D_{ik}$ iff $\{ j \in J \: | \: f_U(j) \in {D_{ik}}_j \} \in U$.
\end{enumerate}
where $f_U$ and $g_U$ are equivalence classes of $f$ and $g$ modulo $U$.

See~\cite[Lemma 3.5]{goldblatt1995elementary}, 
a similar construction for modal logics could be found in~\cite{fine1975some}:
\begin{lemma}~\label{ultrapmor}

Let ${(F_j)}_{j \in J}$ be an indexed family of full Cartesian structures of dimension 
$\omega$ and $U$ an ultrafilter on $J$. There exists a $p$-morphism:

\begin{center}
$\varphi : \prod \limits_{J} F_j / U \twoheadrightarrow \mathfrak{S}_{\omega}((\prod \limits_{J} W_j / U))$
\end{center}
that restricts to an isomorphism $F^x \cong I^{\varphi(x)}$ of generated subframes generated by $x \in F$.
\end{lemma}

\begin{proof}
Consider the equation:
\begin{equation} \label{func}
f_i(j) = f{(j)}_i
\end{equation}
If $j \in \prod \limits_{j \in J} W^{\omega}_{j}$, 
then the equation defines a function $f_i \in \prod_{j \in J} W_j$ for each $i < \omega$. 
Then a sequence ${(f_i)}_{i < \omega}$ defines a function by Equation~\ref{func}. 
Clearly $f_U = g_U$ implies ${f_i}_U = {g_i}_U$ for $i < \omega$. So define $\varphi$ as:
\begin{equation}
\varphi(f_U) = {({f_i}_U)}_{i < \omega}
\end{equation}

It is readily checked that:
\begin{enumerate}
\item $f_U \in D_{k l}$ iff ${f_k}_U = {f_l}_U$ iff $\varphi(f_U) \in E^{\omega}_{k l}$,
\item $(f_U) R_k (g_U)$ implies ${f_l}_U = {g_l}_U$ 
whenever $k \neq l < \omega$, so $(f_U) R^{\omega}_k (g_U)$, so $\varphi$ is monotone.
\end{enumerate}
Let us show that $\varphi$ has the lifting property. Assume that $\varphi(f_U) R^{\omega}_k z$ 
where $z = {(g_k)}_{k < \omega}$. We have got to show that there exists $h_U$ such that 
$\varphi(h_U) = z$ and $(f_U) R_k (h_U)$. Put $h_k = g_k$ and $h_l = f_l$ for $k \neq l < \omega$, 
so for $k \neq l$ one has $P{(f_l)}_U = {(g_l)}_U$ since $\varphi(f_U) R^{\omega}_k z$, 
so ${(g_l)}_U = {(h_l)}_U$, so $z = (h_U)$ are the same sequence. 
Moreover, $\{ j \: | \: h(j) {R_k}_j f(j)\} = J \in U$, 
since ${h(j)}_l = {f(j)}_l$ for $l \neq k$, so $(f_U) R_k (h_U)$ in the ultraproduct.

\vspace{\baselineskip}

Let us show that $\varphi$ acts isomorphically on every generated subframe $F^{x}$ of the ultraproduct. Take $f_U, g_U \in F^{x}$, then there are $i_0, \dots, i_n < \omega$ such that
\begin{center}
$f_U(R_{i_0} \circ \dots \circ R_{i_n})g_U$.
\end{center}
By \strokeL o\'s's theorem we have
\begin{center}
$J_{fg} = \{ j \in J \: | \: f(j)(R_{{i_0}_j} \circ \dots \circ R_{{i_n}_j})g(j) \}$
\end{center}
So for $J_{fg}$, the $\omega$-sequences $f(j)$ and $g(j)$ agree except possibly on $i_0, \dots, i_n$. 
If $\varphi(f_U) = \varphi(g_U)$, then for each $k < \omega$, $f_{k_U} = g_{k_U}$ and then:
\begin{center}
$J_k = \{ j \in J \: | \: f_k(j) = g_k(j) \} \in U$
\end{center}
But $f, g$ are identical on the set
\begin{center}
$J_k \cap J_{i_0} \cap \dots \cap J_{i_n} \in F$
\end{center}
and thus $f_U = g_U$, so $\varphi$ is injective on $F^{x}$.
\end{proof}

\begin{theorem} \label{puub}
${\bf Pu}\mathcal{F}\mathfrak{ct}_{\omega} \subseteq {\bf Ub}\mathcal{F}\mathfrak{ct}_{\omega}$.
\end{theorem}

\begin{proof}
Let $F = \prod \limits_J F_j / U$ be an ultraproduct of full Cartesian structures of dimension $\omega$. 
To show $F \in {\bf Ub}\mathcal{F}\mathfrak{ct}_{\omega}$ one needs to show that for each point $x \in F$ 
there exists a generated subframe that contains $x$ and is isomorphic to 
$I = \mathfrak{S}_{\omega}((\prod \limits_J F_j / U))$.

Let $Z$ be a choice set that contains exactly one element from each weak Cartesian substructure of $I$. 
But $I$ is the disjoint union of all its weak substructures, so we have:
\begin{center}
$I = \coprod \limits_{z \in Z} I^z$
\end{center}
Fix $x \in F$, for each $z \in Z$ choose $\psi(z)$ to be any member of $F$ such that 
$\varphi(\psi(z)) = z$ and $I^z$ is the weak substructure containing $\varphi(x)$, 
where $\varphi$ is a $p$-morphism from Lemma~\ref{ultrapmor}. By the previous lemma, we have
\begin{center}
$F^{\psi(z)} = I^z$.
\end{center}

If $z$ and $z'$ are different elements of $Z$, so $I^z$ and $I^{z'}$ are disjoint, 
so $F^{\psi(z)}$ and $F^{\psi(z')}$ are also disjoint. 

$F(x)$ is defined to be the union of the collection of $\{ F^{\psi(z)} \: | \: z \in Z \}$ and 
forms a generated subframe of $F$ which is isomorphic of $I^z$'s, so $F^x \cong I$,
but $x = \psi(z)$ for some $z$, so $x \in F(x)$. 
\end{proof}

\begin{col}
${\bf Ub}\mathcal{F}\mathfrak{ct}_{\omega}$ is closed under ultraproducts.
\end{col}

\begin{theorem}
$ $

\begin{enumerate}
\item ${\bf Pu}\mathcal{W}\mathfrak{ct}_{\omega} \subseteq {\bf Ub}\mathcal{W}\mathfrak{ct}_{\omega}$,
\item ${\bf Pu}\mathcal{S}\mathfrak{ct}_{\omega} \subseteq \mathcal{S}\mathfrak{ct}_{\omega}$.
\end{enumerate}
\end{theorem}

\begin{proof}
$ $
\begin{enumerate}
\item Let $F^* = \prod \limits_{J} F^*_j / U$ be an ultraproduct of weak Cartesian structures of dimension $\omega$. 
Each $F^*_j$ is a generated subframe of some full Cartesian structure $F_j$, 
so $F^*$ is isomorphic to a generated subframe of the ultraproduct 
$F = \prod \limits_{J} F_j / U$ and we identify $F^*$ with this generated subframe.
But the ultraproduct $F$ is a cylindric $\omega$-frame since each $F_j \in {\bf CFrames}_{\omega}$,
but ${\bf CFrames}_{\omega}$ is elementary and then closed under ultraproducts. 
But $F^*$ can be defined as
\begin{center}
$F^* = \coprod \limits_{x \in F^*} F^x$
\end{center}
But each $F^x$ is isomorphic to each $I^{\varphi(x)}$.
\item Let $F^* = \prod \limits_{J} F^*_j / U$ be an ultraproduct of sub-Cartesian structures of dimension 
$\omega$, then each $F^*_j$ is a substructure of some full Cartesian structure $F_j$ of dimension $\omega$.
So $F^*$ is isomorphic to a substructure of the ultraproduct $F = \prod \limits_{J} F_j / U$. 
As in the previous item, $F \in {\bf CFrames}_{\omega}$ of its all point-generated substructures $F^x$,
each of which is isomorphic to some sub-Cartesian structure of dimension $\omega$. Then
\begin{center}
$F = \coprod \limits_{x \in F} F^{x} \in {\bf Ud}\mathcal{S}\mathfrak{ct}_{\omega} = \mathcal{S}\mathfrak{ct}_{\omega}$
\end{center}
That makes $F^* \in \mathcal{S}\mathfrak{ct}_{\omega}$.
\end{enumerate}
\end{proof}

\begin{theorem}~\label{cmfact}
$\mathcal{W}\mathfrak{ct}_{\omega}^{+} \subseteq {\bf S} ({({\bf Pw}(\mathcal{W}\mathfrak{ct}_{\omega}))}^{+})$.
\end{theorem}

\begin{proof}
Let $J$ be the set of finite subsets $\omega$ and let $U$ be an ultrafilter on $J$ that contains, 
for each $i \in J$, the set
\begin{center}
$J_i = \{ j \in J \: | \: i \subseteq j \}$.
\end{center}
In particular $J_k = \{ j \in J \: | \: k \in j \}$ for all $k < \omega$. 
Now take $\mathfrak{S}_{\omega}(X^{(x)}) \in \mathcal{W}\mathfrak{ct}_{\omega}$.
For each $y \in {X^{\omega}}^{(x)}$, let $f_y \in ({X^{\omega}}^{(x)})^{J}$ be the constant 
function $f_y(j) = y$. Then $\psi : y \mapsto f_y / U$ is the isomorphic embedding
\begin{center}
$\psi : \mathfrak{S}_{\omega}(X^{(x)}) \hookrightarrow \mathfrak{S}_{\omega}{(X^{(x)})}^J/U$
\end{center}
of $\mathfrak{S}_{\omega}(X^{(x)})$ to its ultrapower $\mathfrak{S}_{\omega}(X^{(x)})^J/U$ with respect to $U$. We have got to show:
\begin{claim} \label{claim}
There exists a bounded morphism $\varphi : \mathfrak{S}_{\omega}{(X)}^J/U \to \mathfrak{S}_{\omega}(X^{(x)})/U$ 
such that its image contains the image of $\psi$:
\begin{center}
$\psi : \mathfrak{S}_{\omega}(X^{(x)}) \hookrightarrow \operatorname{Im}\psi \subseteq \operatorname{Im}\phi \subseteq \mathfrak{S}_{\omega}{(X^{(x)})}^J / U $
\end{center}
\end{claim}

By duality $\phi$ induces a homomorphism:
\begin{center}
$\phi^+ : {(\mathfrak{S}_{\omega}(X^{(x)})/U)}^+ \to {(\mathfrak{S}_{\omega}{(X)}^J/U)}^+$
\end{center}
$\phi^+$ composes with the homomorphism:
\begin{center}
${(\mathfrak{S}_{\omega}(X^{(x)}))}^+ \to {(\mathfrak{S}_{\omega}{(X^{(x)})}^J/U)}^+$
\end{center}
that gives a homomorphism:
\begin{center}
$\theta : {(\mathfrak{S}_{\omega}(X^{(x)}))}^+ \to {(\mathfrak{S}_{\omega}{(X)}^J/U)}^+$
\end{center}
Let us describe the action of $\theta$, take $f \in {(X^{\omega})}^J$ and choose any 
$f^{\bullet} \in {({X^{\omega}}^{(x)})}^J$ such that $\varphi(f/U) = f^{\bullet}/U$, 
so for any $Y \subseteq {X^{\omega}}^{(x)}$:
\begin{center}
$\theta(Y) = \{ f/U \in {(X^{\omega})}^J/U \: | \: \{ j \: | \: f^{\bullet}(j) \in Y \} \in U \}$.
\end{center}

So for $y \in Y$, then $\psi(y)$ (that is $f_y/U$) is equal to $\varphi(f/U)$ for some $f$, 
so then $f^{\bullet}/U = f_y/U$ and then $\{ j \: | \: f^{\bullet}(j) = y \in Y \} \in U$ 
showing that $f/U \in \theta(Y)$. 
As far as $\mathfrak{S}_{\omega}{(X)}^J/U$ is an ultrapower of a full Cartesian structure of dimension 
$\omega$, so ${(\mathfrak{S}_{\omega}(X^{(x)}))}^+ \in {\bf S} ({({\bf Pw}(\mathcal{F}\mathfrak{ct}_{\omega}))}^+)$, so the theorem is proved.

Now let us prove Claim~\ref{claim}:
\begin{proof}
Take $f \in (X^{\omega})^J$, define $f^{\bullet} \in {({X^{\omega}}^{(x)})}^J$ as:
\begin{equation}
    f^{\bullet}(j)_k = \begin{cases}
        {f(j)}_k \:\: \text{if $k \in j$} \\
        x_k \:\: \text{otherwise}
    \end{cases}
\end{equation}
Each $f^{\bullet}(j)$ differs from $x$ at most on the finite set $j$. Clearly that $f(j) = g(j)$ implies $f^{\bullet}(j) = g^{\bullet}(j)$, 
so $f_U = g_U$ in ${(X^{\omega})}^J/U$ implies $f^{\bullet}_U = g^{\bullet}_U$ in 
${({X^{\omega}}^{(x)})}^J/U$. So the mapping $\varphi : f_U \mapsto f^{\bullet}_U$ is well-defined.

Let us show that $\operatorname{Im}\psi \subseteq \operatorname{Im}\varphi$. 
Take ${f_y}_U \in \operatorname{Im}\psi$ with $y \in {X^{\omega}}^{(x)}$. 
We also have ${f_y}_U \in {(X^{\omega})}^J$, so that is enough to show that 
${f^{\bullet}_y}_U = {f_y}_U$ in $\operatorname{Im}\psi$. Put $i = \{ k < \omega \: | \: x_k \neq y_k \} \in J$, 
so for $j \in J_j$:
\begin{equation}
    {f^{\bullet}_y(j)}_k = \begin{cases}
        {f_y(j)}_k \:\:\:\:\:\:\:\:\:\:\:\: \text{if $k \in j$} \\
        x_k = {f_y(j)}_k \:\: \text{otherwise}
    \end{cases}
\end{equation}
since $f_y(j)$ agrees with $x$ outside $i$. Thus $f^{\bullet}_y(j) = f_y(j)$ for each $j \in J_j \in U$,
so ${f^{\bullet}_y}_U = {f_y}_U = \psi(y)$.
\end{proof}

We skip the proof $\varphi$ is a bounded morphism. 
\end{proof}

\begin{theorem}
$\mathcal{W}\mathfrak{ct}_{\omega}^+ \subseteq {\bf RCA}_{\omega}$.
\end{theorem}

\begin{proof}
$ $

$\begin{array}{lll}
\mathcal{W}\mathfrak{ct}_{\omega}^+ \subseteq & \\
\:\:\:\: \text{By Theorem~\ref{cmfact}} & \\
\:\:\:\:\:\:\:\:\:\:\:\:\:\:\:\:\:\:\:\:\:\:\:\:\:\:\:\:\:\:\: \subseteq {\bf S Cm Pw}(\mathcal{F}\mathfrak{ct}_{\omega}) & \\
\:\:\:\: \text{By Theorem~\ref{puub}} & \\
\:\:\:\:\:\:\:\:\:\:\:\:\:\:\:\:\:\:\:\:\:\:\:\:\:\:\:\:\:\:\: \subseteq {\bf S Cm Ub}(\mathcal{F}\mathfrak{ct}_{\omega}) & \\
\:\:\:\: \text{Since ${\bf Ub} \leq \mathbb{H}{\bf Ud}$} & \\
\:\:\:\:\:\:\:\:\:\:\:\:\:\:\:\:\:\:\:\:\:\:\:\:\:\:\:\:\:\:\: \subseteq {\bf S Cm} \mathbb{H}{\bf Ud}\mathcal{F}\mathfrak{ct}_{\omega} & \\
\:\:\:\: \text{Since $\mathbb{H}{\bf Ud} = {\bf Ud}$} & \\
\:\:\:\:\:\:\:\:\:\:\:\:\:\:\:\:\:\:\:\:\:\:\:\:\:\:\:\:\:\:\: = {\bf S Cm} {\bf Ud}\mathcal{F}\mathfrak{ct}_{\omega} & \\
\:\:\:\:\:\:\:\:\:\:\:\:\:\:\:\:\:\:\:\:\:\:\:\:\:\:\:\:\:\:\: = {\bf RCA}_{\omega}& \\
\end{array}$
\end{proof}

\begin{theorem}
${\bf IWs}_{\omega} \subseteq {\bf ICs}_{\omega}$.
\end{theorem}

\begin{proof}
    
\end{proof}

\subsection{The proof of canonicity}

Here we use the following fact related to canonical varieties generated by some class of complex algebras. Let ${\bf K}$ be an elementary class of relational structures, then:
\begin{center}
If $\mathcal{K}$ is closed under $p$-morphic images, generated subframes, and disjoint unions, then ${\bf S} (\mathcal{K}^+)$ is a canonical variety.
\end{center}
One may think of this fact as a more abstract version of Fine's theorem which claims that every elementary modal logic is canonical \cite{fine1975some}. This version denotes the same fact, but it is formulated in terms of varieties BAOs generated by complex algebras of some atom structures. We provide a more precise formulation of the fact above.


\begin{prop}
Let $\mathcal{K}$ be a class of frames, then ${\bf Pu}\mathcal{K} \subseteq \mathbb{H}\mathbb{S}{\bf Ud} \mathcal{K}$ implies that ${\bf S}((\mathbb{S}{\bf Ud} \mathcal{K})^+)$ is a canonical variety.
\end{prop}

This is a special case of \cite[Theorem 4.4]{goldblatt1995elementary} for dimension $\omega$.

\begin{theorem}
${\bf RCA}_{\omega}$ is a canonical variety.
\end{theorem}

\begin{proof}
We have ${\bf RCA}_{\omega} = {\bf S}((\mathbb{S} {\bf Ud} \mathcal{F}\mathfrak{ct}_{\omega})^+)$.
That's enough to show that ${\bf Pu} \mathcal{F}\mathfrak{ct}_{\omega} \subseteq \mathbb{HS} {\bf Ud} \mathcal{F}\mathfrak{ct}_{\omega}$.
For that, we need the following claim:

\end{proof}

\section{Representability via games}

\subsection{Monk's theorem for ${\bf RCA}_n$ via saturation}

In this section we consider classes ${\bf RCA}_n$ for $n < \omega$.

We provide the complete proof of the following theorem \cite[Theorem 3.4.3]{hirsch2013completions}.
\begin{theorem}\label{finitecanon}
Let $\mathcal{A} \in {\bf CA}_n$, then $\mathcal{A}$ is representable iff ${(\mathcal{A}_+)}^{+}$ is completely representable.
\end{theorem}

For that we need such model-theoretic notions as saturation and types, see~\cite[Section 6.3]{hodges1993model}.

\begin{definition} Let $\mathcal{M}$ be a first-order structure of a signature $L$ and $S \subseteq \mathcal{M}$. Let $L(S)$ be an extension of $L$ with copies of elements from $S$ as additional constants. We assume that $Cnst(L)$ and $S$ are disjoint.

\begin{enumerate}
\item Let $n < \omega$, an $n$-type over $S$ is a set $\mathcal{T}$ of $L(S)$ formulas $A(\overline{x})$, where $\overline{x}$ is a fixed $n$-tuple of elements from $S$. Notation: $\mathcal{T}(\overline{x})$. A type is an $n$-type for some $n < \omega$.
\item An $n$-type $\mathcal{T}(\overline{x})$ is realised in $\mathcal{M}$, if there exists $\overline{m} \in \mathcal{M}^n$ such that $\mathcal{M} \models A(\overline{m})$ for every $A \in \mathcal{T}(\overline{x})$. $\mathcal{M}$ omits $\mathcal{T}(\overline{x})$, if $\mathcal{T}(\overline{x})$ is not realised in $\mathcal{M}$.
\item $\mathcal{T}(\overline{x})$ is finitely satisfied in $\mathcal{M}$, if every finite subtype $\mathcal{T}_0(\overline{x}) \subseteq \mathcal{T}(\overline{x})$ is realised in $\mathcal{M}$. We can reformulate that as $\mathcal{M} \models \exists \overline{a} \bigwedge \limits_{A \in \mathcal{T}_0} A(\overline{a})$.
\item Let $T$ be a theory, then a type $\mathcal{T}$ over the empty set of constants is $T$-consistent, if there exists a model $\mathcal{M} \models T$ such that $\mathcal{T}$ is finitely satisfied in $\mathcal{M}$.
\item Let $\kappa$ be a cardinal, then $\mathcal{M}$ is $\kappa$-saturated, if for every $S \subseteq \mathcal{M}$ with $|S| < \kappa$ every finitely satisfied $1$-type $\mathcal{T}$ is realised in $\mathcal{M}$.
\end{enumerate}
\end{definition}

By default, a saturated model is an $\omega$-saturated model for us.

A couple of useful facts from \cite{chang1990model} and \cite{hodges1993model}:

\begin{fact} Let $\mathcal{M}$ be an FO-structue and $\kappa$ a cardinal, then:
\begin{enumerate}
\item $\mathcal{M}$ is $\kappa$-saturated, iff every finitely satisfiable $\alpha$-type (an arbitrary $\alpha \leq \kappa$) with fewer than $\kappa$ parameters is realised in  $\mathcal{M}$.
\item If $\mathcal{M}$ is $\kappa$-saturated, then $\mathcal{M}$ is $\lambda$-saturated for every $\lambda < \kappa$.
\item \label{saturation} Every consistent theory has a $\kappa$-saturated model and every model has an elementary $\kappa$-saturated extension.
\item Let $(\mathcal{M}_i)_{i < \omega}$ a family of structures of the (at most) countable signature and $D$ a non-principal ultrafilter over $\omega$, then $\Pi_D \mathcal{M}_i$ is $\omega_1$-saturated.
\end{enumerate}
\end{fact}

\subsection{Proof of Theorem~\ref{finitecanon}}

Let $\mathcal{A} \in {\bf CA}_n$, then if $\mathcal{A}$ is completely representable, then $h$, a complete representation of $\mathcal{A}$, is atomic. That is, $(a_1, \dots, a_n) \in h(1)$, then $(a_1, \dots, a_n) \in h(y)$ for some $y \in \operatorname{At}(\mathcal{A})$.

\begin{definition} \label{theory} Let $\mathcal{A}$ be a cylindric algebra of dimension $n < \omega$.
$L(\mathcal{A})$ is the first-order language that consists of equality plus $n$-ary predicate letters $(R^n_a)_{a \in \mathcal{A}}$. The $L(\mathcal{A})$-theory $T_{\mathcal{A}}$ consists of the following sentences:
\begin{enumerate}
\item $A_+(a,b,c) := \forall x_1, \dots, x_n \: (R_a(x_1, \dots, x_n) \leftrightarrow R_b(x_1, \dots, x_n) \lor R_c(x_1, \dots, x_n))$. Informally, that means $\mathcal{A} \models a = b + c$.
\item $A_{-}(a,b) := \forall x_1, \dots, x_n \: (R_a(x_1, \dots, x_n) \leftrightarrow \neg R_b(x_1, \dots, x_n))$. That is, $\mathcal{A} \models a = - b$.
\item $A_{\neq 0}(a) := \exists x_1, \dots, x_n R_a(x_1, \dots, x_n)$. That is, $\mathcal{A} \models a \neq 0$.
\item $A_{c_i}(a) := \forall x_1, \dots, x_n (R_{c_i a}(x_1, \dots, x_n) \leftrightarrow \exists y_1, \dots y_n (R_a(y_1, \dots, y_n) \land x_i = y_j))$, for $i < n$ and $j < n$ such that $i \neq j$. Informally, $\mathcal{A} \models {c_i} a = 1$.
\item $A_{d_{ij}} := \forall x_1, \dots, x_n (R_{d_{ij}}(x_1, \dots, x_n) \leftrightarrow x_i = x_j)$, for $i, j < n$.
\end{enumerate}
\end{definition}

In fact, we need to show the following implication:
\begin{center}
If $\mathcal{A}$ is representable, then $A^{+}$ is completely representable.
\end{center}

Assume that $\mathcal{A}$ is representable, then the theory $T(\mathcal{A})$ is consistent, then it has an $\omega$-saturated model $\mathcal{M}$ by Fact~\ref{saturation}. We have the following claim:
\begin{claim}
The set $U_{x_1,\dots,x_n} = \{ a \in \mathcal{A} \: | \: \mathcal{M} \models R_a(x_1,\dots,x_n)\}$ is an ultrafilter of $\mathcal{A}$, for $x_1,\dots,x_n \in \mathcal{M}$ with $R_{1}(x_1,\dots,x_n)$.
\end{claim}
Those $U_{x_1,\dots,x_n}$'s allow us to represent atoms of $\mathcal{A}^{+}$.

We define a representation of $\mathcal{A}^{+}$ as a map $h : \mathcal{A}^{+} \to 2^{\mathcal{M}^n}$ such that:
\begin{center}
$h : S \mapsto \{ (x_1, \dots, x_n) \in 1^{\mathcal{M}} \: | \: U_{x_1,\dots,x_n} \in S \}$, for $S \in {\bf Spec}(\mathcal{A})$.
\end{center}

\begin{claim} Let $A_1, A_2 \in {\bf Spec}(\mathcal{A})$
\begin{enumerate}
\item $h(0^{\mathcal{A}^{+}}) = \emptyset$
\item $h(- A_1) = - h(A_1)$
\item $h(1^{\mathcal{A}^{+}}) = 1^{\mathcal{M}}$
\item If $S \subseteq {\bf Spec}(\mathcal{A})$, then $h(\bigcup S) = \bigcup \limits_{U \in S} h(U)$
\end{enumerate}
In particular, $h$ is a Boolean homomorphism.
\end{claim}

\begin{proof}
$ $

\begin{enumerate}
\item $h(0^{\mathcal{A}^{+}}) = h(\emptyset) = \emptyset$.
\item From the definiton of $h$.
\item $h(- A_1) = - h(A_1)$

Let $x_1, \dots, x_n \in 1^{\mathcal{M}}$, then we have:

\begin{center}
$(x_1, \dots, x_n) \in h(- A_1)$ iff $U_{x_1, \dots, x_n} \in - A_1$ iff $U_{x_1, \dots, x_n} \notin A_1$ iff $(x_1, \dots, x_n) \notin h(A_1)$
\end{center}
\item Let $S = \bigcup \limits_{i \in I} S_i$, where $S_i \in {\bf Spec}(\mathcal{A})$ for every $i \in I$.
Let $(x_1, \dots, x_n) \in 1^{\mathcal{M}}$, then we have:
\begin{center}
$(x_1, \dots, x_n) \in h(\bigcup \limits_{i \in I} S_i)$ iff $f_{x_1, \dots, x_n} \in \bigcup \limits_{i \in I} S_i$ iff $\exists i \in I \:\: f_{x_1, \dots, x_n} \in S_i$ iff $\exists i \in I \:\: (x_1, \dots, x_n) \in h(S_i)$ iff $(x_1, \dots, x_n) \in \bigcup \limits_{i \in I} S_i$
\end{center}
\end{enumerate}
\end{proof}

\begin{claim}
$h$ is injective.
\end{claim}

\begin{proof}
Let $U \in {\bf Spec}(\mathcal{A})$. The first is to show that $h(U)$ is non-empty. The following $n$-type:
\begin{center}
$T(x_1, \dots, x_n) = \{ R_a(x_1, \dots, x_n) \: | \: a \in U \}$
\end{center}
if finitely satisfied in $\mathcal{M}$.

Consider $T_0 = \{ R_{a_1}(x_1, \dots, x_n), \dots, R_{a_k}(x_1, \dots, x_n) \} \subseteq T$. Then $a_1, \dots, a_k \in U$ and $a = a_1 \cdot \dots \cdot a_k \in U$. By the instance of the $A_{\neq 0}(a)$-axiom, we have $\mathcal{M} \models \exists x_1, \dots, x_n R_a(x_1, \dots, x_n)$.
$a \leq a_i$ for $i \leq k$, so we have $\mathcal{M} \models \exists x_1, \dots, x_n R_{a_i}(x_1, \dots, x_n)$ for every $a_i$ with $i \leq k$ by the instance of the $A_{+}(a_i, a, a)$-axiom. That makes every finite subtype of $T$ satisfiable, thus the whole type is finitely satisfiable in $\mathcal{M}$.
$\mathcal{M}$ is $\omega$-saturated, then $T$ is realised in $\mathcal{M}$ by some $(x_1, \dots, x_n) \in \mathcal{M}^n$ and, moreover, $\mathcal{M} \models 1(x_1, \dots, x_n)$. As we have already said, $U_{x_1, \dots, x_n}$ is an ultrafilter, but $U_{x_1, \dots, x_n} \subseteq U$, thus $U = U_{x_1, \dots, x_n}$, so $(x_1, \dots, x_n) \in h(U)$.

That makes $h$ one-to-one.
\end{proof}

\begin{claim}
$ $

\begin{enumerate}
\item $h(c_i{^{\mathcal{A}^{+}}} U) = C_i (h(U))$
\item $h(d_{ij}^{\mathcal{A}^{+}}) = D_{ij} \subseteq {\bf Spec}(\mathcal{A})$
\end{enumerate}
\end{claim}

\begin{proof}
$ $

\begin{enumerate}
\item Assume $(x_1, \dots, x_n) \in h(c_i{^{\mathcal{A}^{+}}} S)$.

Let us show that $\overline{x} \in C_i (h(S))$, that is, there exists $\overline{y} = (y_1, \dots, y_n) \in h(S)$ such that $\overline{x} \equiv_{i} \overline{y}$.

Then $\mathcal{M} \models 1(x_1, \dots, x_n)$ and $U_{x_1, \dots x_n} \in c_i{^{\mathcal{A}^{+}}} S$. But $\mathcal{A}^{+}$ is the complex algebra of the ultrafilter frame $\mathcal{F}_{\mathcal{A}}$.
Then we have:
\begin{center}
$c_i{^{\mathcal{A}^{+}}} S = \{ U_1 \in {\bf Spec}(\mathcal{A}) \: | \: \exists U' \in S  \: U_1 R_{i} U' \}$
\end{center}
Then there must be an ultrafilter $U' \in S$ such that $U_{x_1, \dots x_n} R_{i} U'$, that is, $c_i a \in U_{x_1, \dots x_n}$ whenever $a \in U'$.
Hence $\mathcal{M} \models R_{c_i}(x_1, \dots x_n)$. By the $A_{c_i}(a)$-axiom, we have
\begin{center}
$\mathcal{M} \models \exists z_1, \dots, z_n (R_a(z_1, \dots, z_n) \land x_i = z_j)$ for $i < n$ and $j < n$ such that $i \neq j$.
\end{center}
Consider the following $n$-type with free variables $z_1, \dots, z_n$ and parameters $x_1, \dots, x_n \in \mathcal{M}$:
\begin{center}
$T(z_1, \dots, z_n) = \{ R_a(z_1, \dots, z_n) \land x_i = z_j \: | \: i < n, j < n, i \neq j, a \in U' \}$.
\end{center}
Let us show that $T(z_1, \dots, z_n)$ is finitely satifsiable in $\mathcal{M}$.
Consider a finite subset of $T$, say $T_0 = \{ R_{b_k}(z_1, \dots, z_n) \land x_i = y_j \: | \: i < n, j < n, i \neq j, b_k \in U', k < \omega \}$.
We put $p = p_1 \cdot \cdots \cdot p_k$ and $p \in U'$ since $U'$ is a filter. Then we have:
\begin{center}
$\mathcal{M} \models \exists z_1, \dots, z_n (R_b(z_1, \dots, z_n) \land x_i = z_j)$ for $i < n$ and $j < n$ such that $i \neq j$
\end{center}
Thus, we have, as required:
\begin{center}
$\mathcal{M} \models \exists z_1, \dots, z_n \bigwedge \limits_{i = 1}^{k} (R_{b_k}(z_1, \dots, z_n) \land x_i = z_j)$ for $i < n$ and $j < n$ such that $i \neq j$.
\end{center}
As above, using $\omega$-saturation, we conclude that $T$ is realised in $\mathcal{M}$ at an $n$-tuple $(y_1, \dots, y_n) = \overline{y}$.
Then we have:
\begin{center}
$\mathcal{M} \models 1(\overline{y})$, $\overline{x} \equiv_i \overline{y}$, $U_{\overline{y}} \supseteq U'$
\end{center}
Then $U_{\overline{y}} = U'$, then $\overline{y} \in h(S)$. Then $\overline{x} \in C_i (h (S))$.

Suppose for the converse, $\overline{x} = (x_1, \dots, x_n) \in C_i (h(S))$. We need $\overline{x} \in h(c_i(S))$.
Then there exists $\overline{y} = (y_1, \dots, y_n)$ such that $\overline{x} \equiv_i \overline{y}$ and $\overline{y} \in h(S)$. Then there exists an ultrafilter $U_{y_1, \dots, y_n} \in S$. Let us show that $\mathcal{M} \models 1(x_1, \dots, x_n)$ and $U_{x_1, \dots, x_n} \in c_i U_{y_1, \dots, y_n}$.
Let $a \in U_{y_1, \dots, y_n}$. Then we have $\mathcal{M} \models R_a(y_1, \models, y_n)$. By the $A_{c_i}(a)$ axiom, we have $\mathcal{M} \models R_{c_i a}(x_1, \dots, x_n)$. Then $\mathcal{M} \models 1(x_1, \dots, x_n)$ and $c_i a \in U_{x_1, \dots, x_n}$, thus, $\overline{x} \in h(c_i(S))$.
\item Let us show that $h$ preserves cylindrifications.

Let $(x_1, \dots, x_n) \in \mathcal{M}^n$. Then $(x_1, \dots, x_n) \in D_{ij}$ iff $\mathcal{M} \models 1(x_1, \dots, x_n)$ and $x_i = x_j$ iff $U_{x_1, \dots, x_n} \in d_{ij}^{\mathcal{A}^{+}} = \{ U \in {\bf Spec}(\mathcal{A}) \: | \: d_ij \in U \}$ iff $\mathcal{M} \models d_{ij}^{\mathcal{M}}(x_1, \dots, x_n)$.
\end{enumerate}
\end{proof}

\subsection{Finite dimensions}

\begin{definition}

Network
\end{definition}

\begin{theorem}

Completely representable iff $\exists$ has a ws.
\end{theorem}

\begin{definition}
Ultrafilter network
\end{definition}

\begin{theorem}
${\bf RCA}_n$ is a pseudoelementary class for $3 \leq n < \omega$.
\end{theorem}

\begin{theorem}

$\exists$ has a ws for the canonical extension.

\end{theorem}

\subsection{Dimension $\omega$}

\begin{question}
Can we characterise ${\bf RCA}_{\omega}$ as an enumerably axiomatisable pseudo-elementary class in three-sorted logic with sorts ${\bf b}$ (Boolean part), ${\bf c}$ (cylindric part) and ${\bf r}$ (representation part)?
\end{question}


\begin{definition}
Network
\end{definition}

\begin{theorem}

Completely representable iff $\exists$ has a ws.
\end{theorem}

\begin{definition}
Ultrafilter network
\end{definition}

\begin{theorem}

$\exists$ has a ws for the canonical extension.

\end{theorem}

\subsection{Counterexamples}

\bibliographystyle{alpha}
\bibliography{Text}

\end{document}
